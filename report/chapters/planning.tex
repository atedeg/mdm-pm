\chapter{Planning}\label{ch:planning}
Osservando i documenti prodotti in precedenza -- come il cartellone dell'event storming e il POS -- il team di Atedeg ha convenuto non fosse necessaria la stesura del \emph{Project Definition Statement}.
Infatti, grazie alla tecnica di event storming, sono state raccolte informazioni di dominio sufficientemente dettagliate a supportare le successive fasi dello sviluppo.

\section{Joint Project Planning Session}\label{sec:JPPS}
Complessivamente si sono svolte tre riunioni di \emph{Joint Project Planning} con il committente. In seguito sono riportati i partecipanti delle riunioni e ciò che è stato deciso opportunamente motivato.

\subsection{Partecipanti}
\begin{itemize}
  \item I membri del core team di Atedeg che hanno già preso parte alle riunioni precedenti: Giacomo, Linda (facilitatore e product owner), Nicolas e Nicolò
  \item Un altro sviluppatore di Atedeg
  \item Raffaella (project sponsor e process owner dei processi relativi alla pianificazione della produzione)
  \item Gianluca (process owner dei processi aziendali relativi a gestione ordini e magazzino)
  \item Simone (process owner dei processi aziendali relativi alla produzione)
\end{itemize}

\subsection{Kick-off}
La prima riunione si avvia con la presentazione dei partecipanti e un breve riassunto di quanto discusso nelle riunioni precedenti. In particolare Raffaella ha spiegato come il progetto si inserisce nella strategia aziendale e quali sono gli obiettivi che si prefigge: efficientare il processo di gestione degli ordini e automatizzare la pianificazione della produzione aumentando la produttività dei macchinari del caseificio.
Dopodiché è avvenuta la presentazione del team di Atedeg.

\subsection{Working session}

\subsubsection{Validazione e panoramica dell'approccio scelto}
Come già discusso nella precedente riunione, il team ha deciso di adottare un approccio di sviluppo Agile con metodologia Scrum. La scelta è stata discussa e approvata con il committente che si è reso disponibile a prendere parte alle riunioni di \emph{sprint review}.

Di comune accordo si è stabilito che la durata di ogni sprint fosse di due settimane e che le riunioni di review si tenessero di venerdì. Infine è stato deciso che la \emph{product owner} sarebbe stata Linda, che ha già esperienza sia con la metodologia Scrum che con lo sviluppo DDD.

\subsubsection{Validazione e prioritizzazione dei requisiti}
Per quanto riguarda i requisiti, il team ha deciso di assegnare insieme al committente una priorità numerica da 1 (la più bassa) a 5 (la più alta) a ciascuna epica.
La stessa scala di priorità è stata usata anche per prioritizzare le user story all'interno della loro rispettiva epica.
Vista l'esigenza del committente di avere un sistema (anche minimo) che permettesse di raccogliere e gestire gli ordini entro settembre, si è deciso di comune accordo di dare la massima priorità all'epica \hyperref[user-story:i2-e4]{I2-E4}, con le altre correlate a seguire.

\subsubsection{Stima delle risorse}
La quantità di lavoro di ciascuna user story è stata stimata assegnandovi un valore scalare adimensionale: gli \emph{story points}.
Per assegnare gli story points è stato usato il metodo \emph{Delphi}: questa tecnica consiste nel chiedere a ciascun membro del team di stimare la quantità di lavoro di una user story, e poi di fare dei nuovi round tenendo conto delle stime e motivazioni degli altri membri del team. Questo procedimento viene ripetuto finché non si converge a un risultato comune o non si raggiunge un numero di round prefissato.
In caso di mancata convergenza, si è deciso di usare la media delle stime come valore finale.

Indicativamente un punteggio di 5-6 story points indica una user story semplice che può essere completata in un paio di giornate lavorative, mentre un punteggio superiore indica user story più complesse che potenzialmente potrebbero richiedere anche l'intero sprint; un esempio di tale user story è la \href{user-story:i1-e1-u1}{I1-E1-U1}.

Una volta stimata la complessità delle user stories è stata adottata nuovamente la Delphi technique per stimare il numero di persone ideale da assegnare a ciascuna storia. Per farlo si è tenuto in considerazione quanto ciascun task potesse essere parallelizzabile cercando di assegnare più sviluppatori a user story complesse.

\subsubsection{Analisi dei rischi}
In seguito sono stati analizzati in dettaglio i rischi riportati nel POS; in particolare, per ciascuno è stata stabilità la probabilità che si verifichi, l'impatto ed eventuali strategie di mitigazione. Il documento risultante è riportato in \Cref{app:analisi-rischi}.

\subsubsection{Gestione del cash flow}
Per stimare i costi del progetto il team fa affidamento alla propria esperienza con progetti passati. Infatti, avendo già esperienza con la realizzazione di portali e-commerce e gestionali il team ha stimato di poter concludere il progetto in 4 sprint di due settimane ciascuno. D'altro canto, la porzione relativa all'integrazione e alla pianificazione della produzione rappresentano la complessità maggiore e si è ipotizzato che possano impiegare 2 interi sprint.
Quindi, nelle stime riportate al cliente viene data come durata indicativa del progetto 6 sprint, rendendo comunque chiaro il fatto che il progetto potrebbe richiedere più tempo per essere terminato.

Ogni dipendente di Atedeg viene pagato \EUR{25} l'ora e la settimana lavorativa è di 32 ore. Per terminare il progetto nei limiti prestabiliti si è scelto di assegnarvi 5 sviluppatori full-time. Dunque, ciascuno sprint avrà un costo di \EUR{8\,500} (tenendo conto anche dei costi delle utenze e delle licenze software) e il progetto complessivamente potrebbe costare indicativamente \EUR{51\,000}.

Il committente ha accettato il progetto con un contratto a corpo dove, al termine di ogni sprint di due settimane viene corrisposta la cifra di \EUR{8\,500}.
Inoltre, il committente ha accettato di pagare un anticipo di \EUR{4\,000} per coprire i costi di avvio del progetto.
