\chapter{Planning}\label{ch:planning}

\section{Joint Project Planning Session}\label{sec:JPPS}
\todo[inline]{Dire quante ne abbiamo fatte, molto probabilmente saranno più di una}

\subsection{Partecipanti}
\todo[inline]{descrivere i partecipanti}
\begin{itemize}
  \item \todo[inline]{tutto il team di atedeg}
\end{itemize}

\subsection{Kick off}
La prima riunione si avvia con la presentazione dei partecipanti e un breve riassunto di quanto discusso nelle riunioni precedenti. In particolare Raffaella ha spiegato come il progetto si inserisce nella strategia aziendale e quali sono gli obiettivi che si prefigge: efficientare il processo di gestione degli ordini e automatizzare la pianificazione della produzione aumentando la produttività dei macchinari del caseificio.
Dopodiché è avvenuta la presentazione del team di Atedeg.

\subsection{Working session}

\subsubsection{Validazione e panoramica dell'approccio scelto}
Come già discusso nella precedente riunione, il team ha deciso di adottare un approccio di sviluppo Agile con metodologia Scrum. La scelta è stata discussa e approvata con il committente che si è reso disponibile a prendere parte alle riunioni di \emph{sprint review}.

Di comune accordo si è stabilito che la durata di ogni sprint fosse di due settimane e che le riunioni di review si tenessero di lunedì. Infine è stato deciso che la \emph{product owner} sarebbe stata Linda, che ha già esperienza sia con la metodologia Scrum che con lo sviluppo DDD.

\subsubsection{Validazione e prioritizzazione dei requisiti}
Per quanto riguarda i requisiti, il team ha deciso di assegnare insieme al committente una priorità numerica da 1 (la più bassa) a 5 (la più alta) a ciascuna epica.
La stessa scala di priorità è stata usata anche per prioritizzare le user story all'interno della loro rispettiva epoca.
Vista l'esigenza del committente di avere un sistema (anche minimo) che permettesse di raccogliere e gestire gli ordini entro settembre, si è deciso di comune accordo di dare la massima priorità all'epica \href{user-story:i2-e4}{I2-E4}, con le altre correlate a seguire.

\subsubsection{Stima della quantità di lavoro}
La quantità di lavoro di ciascuna user story è stata stimata assegnandovi un valore scalare adimensionale: gli \emph{story points}.
Per stimare gli story points è stato usato il metodo \emph{Delphi}: questa tecnica consiste nel chiedere a ciascun membro del team di stimare la quantità di lavoro di una user story, e poi di fare dei nuovi round tenendo conto delle stime e motivazioni degli altri membri del team. Questo procedimento viene ripetuto finché non si converge a un risultato comune o non si raggiunge un numero di round prefissato.
In caso di mancata convergenza, si è deciso di usare la media delle stime come valore finale.

Inoltre, non è stato immediatamente realizzato il \emph{project network diagram} per non imporre una linearizzazione nello sviluppo del progetto che essenzialmente avrebbe fatto perdere i vantaggi nell'utilizzare Scrum. Per questo motivo, si è stabilito di realizzare uno \emph{sprint network diagram} all'inizio di ciascuno sprint per poter permettere al project manager di pianificare al meglio l'uso delle risorse e dei tempi al suo interno.

\subsubsection{Analisi dei rischi}
In seguito sono stati analizzati in dettaglio i rischi riportati nel POS; in particolare, per ciascuno è stata stabilità la probabilità che si verifichi, l'impatto ed eventuali strategie di mitigazione. Il documento risultante è riportato in \Cref{app:analisi-rischi}.

- Working session
  - rischi e piani di mitigazione 
    - possibili assenze di membri del team
    - un membro del team scompare per la conferenza
  - Consenso

