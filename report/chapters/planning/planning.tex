\chapter{Planning}\label{ch:planning}

\section{Joint Project Planning Session}\label{sec:JPPS}
\todo[inline]{Dire quante ne abbiamo fatte, molto probabilmente saranno più di una}

\subsection{Partecipanti}
\todo[inline]{descrivere i partecipanti}
\begin{itemize}
  \item \todo[inline]{tutto il team di atedeg}
\end{itemize}

\subsection{Kick off}
La prima riunione si avvia con la presentazione dei partecipanti e un breve riassunto di quanto discusso nelle riunioni precedenti. In particolare Raffaella ha spiegato come il progetto si inserisce nella strategia aziendale e quali sono gli obiettivi che si prefigge: efficientare il processo di gestione degli ordini e automatizzare la pianificazione della produzione aumentando la produttività dei macchinari del caseificio.
Dopodiché è avvenuta la presentazione del team di Atedeg.

\subsection{Working session}

\subsubsection{Validazione e panoramica dell'approccio scelto}
Come già discusso nella precedente riunione, il team ha deciso di adottare un approccio di sviluppo Agile con metodologia Scrum. La scelta è stata discussa e approvata con il committente che si è reso disponibile a prendere parte alle riunioni di \emph{sprint review}.

Di comune accordo si è stabilito che la durata di ogni sprint fosse di due settimane e che le riunioni di review si tenessero di lunedì. Infine è stato deciso che la \emph{product owner} sarebbe stata Linda, che ha già esperienza sia con la metodologia Scrum che con lo sviluppo DDD.

\subsubsection{Validazione e prioritizzazione dei requisiti}


- Working session
  - Validazione e prioritizzazione dei requisiti
    - Prioritizzazione: dare story points col poker
  - Stima della quantità di lavoro
  - "critical path" consegna in tempo
  - rischi e piani di mitigazione
    - normativa europea che cambia
    - possibili assenze di membri del team
    - un membro del team scompare per la conferenza
  - Consenso

