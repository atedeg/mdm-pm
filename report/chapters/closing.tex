\chapter{Chiusura del progetto}

\section{Deployment}
Si è scelto di effettuare il deployment dei deliverable del progetto con una strategia di tipo ``phased'' e ``parallel''. In particolare, al termine di ogni sprint, in seguito all'accettazione del cliente delle diverse user stories, è stato effettuato il deployment dei sottosistemi realizzati con un approccio di \emph{Continuous Deployment}.
Inoltre, si è prestata particolare attenzione ad affiancare i nuovi sistemi a quelli preesistenti senza effettuare un cut-off decisivo al fine di permettere di verificarne il corretto funzionamento senza correre rischi elevati.

Terminato il progetto e verificata la soddisfazione dei success criteria -- definiti già durante lo scoping -- è stato possibile rimpiazzare definitivamente il precedente sistema con quello realizzato.

\section{Processo di accettazione}
Per assicurarsi il continuo rispetto dei success criteria stabiliti nel Project Overview Statement (\Cref{app:pos}), il team di sviluppo ha deciso, d'accordo con il cliente, che l'accettazione dei deliverable prodotti venga svolta in maniera cadenzata nel tempo.
In particolare, avendo seguito la metodologia Agile, il processo di accettazione da parte del committente è stato effettuato al termine di ciascuno sprint, nella riunione di sprint review.

La metodologia di lavoro scelta ha facilitato il processo di accettazione, grazie anche alla decisione del team di sviluppo di adottare una metodologia di collaudo basata su \emph{Test-Driven Development} e \emph{Continuous Integration}, che prevede la scrittura di test che verificano il corretto funzionamento dei moduli software sviluppati e che vengono eseguiti automaticamente prima di ogni rilascio presso il cliente.

\section{Audit post-implementazione}
Una volta ricevuta l'accettazione da parte del cliente del lavoro svolto durante l'ultimo sprint, è stato svolto un \emph{audit post-implementation}.
Durante questa riunione sono state affrontate diverse tematiche relative all'andamento generale del progetto e a ciò che si sarebbe potuto migliorare o fare diversamente.

In particolare, si è concordato che gli obiettivi del progetto siano stati effettivamente raggiunti: il deliverable consegnato al cliente assolve a tutti i compiti richiesti e soddisfa le aspettative del committente.
Nonostante alcuni imprevisti durante lo sviluppo, si è riusciti a chiudere la parte di progetto responsabile della gestione degli ordini entro i tempi prescritti dal cliente; in ogni caso, l'intero progetto è stato consegnato in circa tre mesi dalla data di inizio dei lavori.
Il progetto ha effettivamente dato valore al cliente, che ora non deve più gestire manualmente gli ordini in ingresso da sorgenti eterogenee e può vantare un piano di produzione che mantenga occupate le macchine più di quanto lo fossero quando questa veniva svolta a mano.

Vista la presenza di una parte sostanziale del progetto che può essere classificata come di ricerca e sviluppo, si è concordato che un approccio Agile sia stato la scelta più opportuna per un'ottimale riuscita del progetto, in quanto ha permesso di avere una maggiore flessibilità, che con approcci più tradizionali sarebbe venuta meno.

Il progetto è stato un'importante occasione di crescita personale per tutti i membri del team, che hanno potuto approfondire diversi concetti innovativi come il Domain-Driven Design e la programmazione funzionale usata in maniera pervasiva.

\section{Chiusura del progetto}
Una volta ricevuta la firma del cliente, per celebrare la conclusione con successo del progetto il team di Atedeg ha organizzato una cena celebrativa (ovviamente a base di formaggi Mambelli).
