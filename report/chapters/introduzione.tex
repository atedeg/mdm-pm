\chapter*{Introduzione}

\emph{Mambelli Domain Model} è un progetto realizzato per il corso di Laboratorio di Sistemi Software.
Il progetto è stato sviluppato da un \href{https://atedeg.dev/about}{gruppo di quattro studenti} per l'azienda cesenate \emph{Mambelli} che si occupa di produzione e vendita di prodotti caseari.

Le modalità di acquisizione dei requisiti, di interazione coi committenti e l'organizzazione tramite metodologia \emph{Agile} riportate nel seguente documento sono state effettivamente impiegate dai membri del gruppo nella realizzazione del progetto. Tuttavia, ai fini dell'esame di \emph{project management} si è immaginato che il team fosse composto da più membri di quanto non fosse in realtà, per poter mettere in luce in maniera più efficace i diversi concetti appresi di project management.

Il progetto è stato sviluppato secondo la metodologia del \emph{Domain Driven Design}~\cite{cit:ddd} facendo ampio uso di \emph{Event Storming}~\cite{cit:event-storming} per gestire la comunicazione con i committenti e la raccolta dei requisiti.

Questo documento contiene tutte le informazioni relative ai processi di project management messi in atto per la gestione del progetto. Inoltre, in appendice sono inclusi tutti i diversi documenti prodotti.
La documentazione tecnica può invece essere consultata all'apposito \href{https://atedeg.dev/mdm/_docs/introduction.html}{sito web}.
