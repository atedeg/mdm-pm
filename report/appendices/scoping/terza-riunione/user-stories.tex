\chapter{User Stories}
\label{app:user-stories}
In seguito sono riportate le user stories definite in accordo con il committente. Le user stories sono state organizzate in iniziative (I) ed epiche (E).
Ciascuna iniziativa fa riferimento ad un diverso bounded context e si compone di epiche e user stories; ciascuna user story è identificata da una sigla che indica l'iniziativa -- ed eventualmente l'epica -- a cui fa riferimento. Per esempio ``I1-E1-U2'' indica la seconda user story dell'epica E1 dell'iniziativa I1.

\section*{I1: Pianificazione della produzione}

\begin{table}[H]
  \begin{tabularx}{\textwidth}{lX}
    \toprule
    \textbf{I1-E1} & \textbf{Generazione del piano di produzione} \\
    \midrule
    \textbf{In quanto} & Raffaella \\
    \textbf{vorrei} & generare automaticamente un piano di produzione per il giorno \\
    \textbf{così da} & ottimizzare l'utilizzo delle macchine per la produzione di formaggio durante la giornata \\
    \midrule
    \textbf{CoS} & Il piano di produzione è adeguato per soddisfare tutti gli ordini in tempo \\
    & Le macchine sono inattive per al massimo il 15\% del tempo \\
    \midrule
    \textbf{Priorità} & 3 \\
    \bottomrule
  \end{tabularx}
  \label{user-story:i1-e1}
\end{table}

\begin{table}[H]
  \begin{tabularx}{\textwidth}{lX}
    \toprule
    \textbf{I1-E1-U1} & \textbf{Soddisfazione degli ordini nel piano di produzione generato} \\
    \midrule
    \textbf{In quanto} & Raffaella \\
    \textbf{vorrei} & verificare che il piano di produzione generato soddisfi tutti gli ordini \\
    \textbf{così da} & assicurarmi che tutti gli ordini possano essere evasi \\
    \midrule
    \textbf{CoS} & Il piano di produzione generato include la produzione di tutte le unità richieste dagli ordini nei tempi prescritti \\
    \midrule
    \textbf{Priorità} & 5 \\
    \textbf{Story points} & 20 \\
    \bottomrule
  \end{tabularx}
  \label{user-story:i1-e1-u1}
\end{table}

\begin{table}[H]
  \begin{tabularx}{\textwidth}{lX}
    \toprule
    \textbf{I1-E1-U2} & \textbf{Ottimizzazione dell'utilizzo delle macchine} \\
    \midrule
    \textbf{In quanto} & Raffaella \\
    \textbf{vorrei} & limitare il tempo di inattività delle macchine per la produzione di formaggio al 15\% o meno \\
    \textbf{così da} & ottenere il massimo rendimento dall'utilizzo delle macchine \\
    \midrule
    \textbf{CoS} & Il piano di produzione generato include solo il tempo necessario per la produzione e le macchine sono inattive per al massimo il 15\% del tempo \\
    \midrule
    \textbf{Priorità} & 3 \\
    \textbf{Story points} & 20 \\
    \bottomrule
  \end{tabularx}
  \label{user-story:i1-e1-u2}
\end{table}

\begin{table}[H]
  \begin{tabularx}{\textwidth}{lX}
    \toprule
    \textbf{I1-E2} & \textbf{Visualizzazione del piano di produzione} \\
    \midrule
    \textbf{In quanto} & Raffaella \\
    \textbf{vorrei} & visualizzare il piano di produzione attuale \\
    \textbf{così da} & conoscere ciò che è stato generato \\
    \midrule
    \textbf{CoS} & L'interfaccia è facile da usare e comprendere secondo le Euristiche di Nielsen~\cite{cit:nielsen} \\
    \midrule
    \textbf{Priorità} & 4 \\
    \bottomrule
  \end{tabularx}
  \label{user-story:i1-e2}
\end{table}

\begin{table}[H]
  \begin{tabularx}{\textwidth}{lX}
    \toprule
    \textbf{I1-E2-U1} & \textbf{Visualizzazione del volume di produzione pianificato} \\
    \midrule
    \textbf{In quanto} & Raffaella \\
    \textbf{vorrei} & visualizzare il numero di unità pianificate per ogni prodotto \\
    \textbf{così da} & conoscere il volume di produzione previsto \\
    \midrule
    \textbf{CoS} & L'interfaccia presenta una tabella con le informazioni sulla pianificazione delle unità di produzione per ogni prodotto \\
    \midrule
    \textbf{Priorità} & 5 \\
    \textbf{Story points} & 1 \\
    \bottomrule
  \end{tabularx}
  \label{user-story:i1-e2-u1}
\end{table}

\begin{table}[H]
  \begin{tabularx}{\textwidth}{lX}
    \toprule
    \textbf{I1-E2-U2} & \textbf{Visualizzazione del volume di produzione effettivo} \\
    \midrule
    \textbf{In quanto} & Raffaella \\
    \textbf{vorrei} & visualizzare il numero di unità effettivamente prodotte per ogni prodotto \\
    \textbf{così da} & confrontare il volume di produzione previsto con quello effettivo \\
    \midrule
    \textbf{CoS} & L'interfaccia presenta una tabella con le informazioni sulla produzione effettiva per ogni prodotto \\
    \midrule
    \textbf{Priorità} & 3 \\
    \textbf{Story points} & 1 \\
    \bottomrule
  \end{tabularx}
  \label{user-story:i1-e2-u2}
\end{table}

\begin{table}[H]
  \begin{tabularx}{\textwidth}{lX}
    \toprule
    \textbf{I1-E2-U3} & \textbf{Filtro nella visualizzazione del piano di produzione} \\
    \midrule
    \textbf{In quanto} & Raffaella \\
    \textbf{vorrei} & filtrare la visualizzazione del piano di produzione in base alla data di pianificazione o di produzione \\
    \textbf{così da} & analizzare il piano di produzione in un determinato periodo di tempo \\
    \midrule
    \textbf{CoS} & L'interfaccia presenta opzioni di filtraggio per la data di pianificazione o di produzione \\
    \midrule
    \textbf{Priorità} & 1 \\
    \textbf{Story points} & 3 \\
    \bottomrule
  \end{tabularx}
  \label{user-story:i1-e2-u3}
\end{table}

\begin{table}[H]
  \begin{tabularx}{\textwidth}{lX}
    \toprule
    \textbf{I1-U1} & \textbf{Modifica del piano di produzione} \\
    \midrule
    \textbf{In quanto} & Raffaella \\
    \textbf{vorrei} & modificare il piano di produzione generato per il giorno \\
    \textbf{così da} & tenere conto di situazioni eccezionali che potrebbero verificarsi durante la giornata \\
    \midrule
    \textbf{CoS} & Il piano di produzione può essere modificato prima di essere approvato per la produzione \\
    \midrule
    \textbf{Priorità} & 5 \\
    \textbf{Story points} & 5 \\
    \bottomrule
  \end{tabularx}
  \label{user-story:i1-u1}
\end{table}

\section*{I2: Ordini dei clienti}

\begin{table}[H]
  \begin{tabularx}{\textwidth}{lX}
    \toprule
    \textbf{I2-U1} & \textbf{Prezzatura automatica} \\
    \midrule
    \textbf{In quanto} & magazziniere \\
    \textbf{vorrei} & che il sistema calcolasse automaticamente i prezzi degli ordini in entrata \\
    \textbf{così da} & evitare di doverlo fare manualmente sfogliando l'elenco dei prezzi \\
    \midrule
    \textbf{CoS} & Il sistema calcola automaticamente i prezzi degli ordini in entrata utilizzando una versione digitale del listino prezzi attualmente in uso \\
    \midrule
    \textbf{Priorità} & 3 \\
    \textbf{Story points} & 4 \\
    \bottomrule
  \end{tabularx}
  \label{user-story:i2-u1}
\end{table}

\begin{table}[H]
  \begin{tabularx}{\textwidth}{lX}
    \toprule
    \textbf{I2-E1} & \textbf{Pallettizzazione automatica} \\
    \midrule
    \textbf{In quanto} & magazziniere \\
    \textbf{vorrei} & pallettizzare un prodotto per un ordine \\
    \textbf{così da} & poter mettere da parte un prodotto per l'ordine di un cliente \\
    \midrule
    \textbf{CoS} & Il magazziniere può scansionare il codice a barre del prodotto con un palmare per contrassegnarlo come pallettizzato e aggiornare le forniture del magazzino \\
    \midrule
    \textbf{Priorità} & 4 \\
    \bottomrule
  \end{tabularx}
  \label{user-story:i2-e1}
\end{table}

\begin{table}[H]
  \begin{tabularx}{\textwidth}{lX}
    \toprule
    \textbf{I2-E1-U1} & \textbf{Contrassegnare il prodotto come pallettizzato} \\
    \midrule
    \textbf{In quanto} & magazziniere \\
    \textbf{vorrei} & contrassegnare il prodotto come pallettizzato \\
    \textbf{così da} & aggiornare le informazioni del magazzino e preparare il prodotto per l'ordine di un cliente \\
    \midrule
    \textbf{CoS} & È possibile interagire con il sistema segnalando la pallettizzazione di un prodotto \\
    \midrule
    \textbf{Priorità} & 5 \\
    \textbf{Story points} & 6 \\
    \bottomrule
  \end{tabularx}
  \label{user-story:i2-e1-u1}
\end{table}

\begin{table}[H]
  \begin{tabularx}{\textwidth}{lX}
    \toprule
    \textbf{I2-E1-U2} & \textbf{Scansione dei prodotti con un palmare} \\
    \midrule
    \textbf{In quanto} & magazziniere \\
    \textbf{vorrei} & contrassegnare il prodotto come pallettizzato scansionandone il codice a barre tramite un palmare \\
    \textbf{così da} & non dover inserire manualmente il codice del prodotto \\
    \midrule
    \textbf{CoS} & L'interfaccia del palmare permette di scansionare il codice a barre del prodotto e contrassegnarlo come pallettizzato \\
    \midrule
    \textbf{Priorità} & 4 \\
    \textbf{Story points} & 6 \\
    \bottomrule
  \end{tabularx}
  \label{user-story:i2-e1-u2}
\end{table}

\begin{table}[H]
  \begin{tabularx}{\textwidth}{lX}
    \toprule
    \textbf{I2-E1-U3} & \textbf{Aggiornamento delle forniture del magazzino} \\
    \midrule
    \textbf{In quanto} & magazziniere \\
    \textbf{vorrei} & aggiornare le forniture del magazzino quando un prodotto viene pallettizzato \\
    \textbf{così da} & tenere traccia della quantità di prodotto disponibile nel magazzino \\
    \midrule
    \textbf{CoS} & Quando un prodotto viene contrassegnato come pallettizzato, le informazioni del magazzino vengono aggiornate di conseguenza \\
    \midrule
    \textbf{Priorità} & 3 \\
    \textbf{Story points} & 5 \\
    \bottomrule
  \end{tabularx}
  \label{user-story:i2-e1-u3}
\end{table}

\begin{table}[H]
  \begin{tabularx}{\textwidth}{lX}
    \toprule
    \textbf{I2-E1-U4} & \textbf{Integrazione con il sistema di tracciabilità} \\
    \midrule
    \textbf{In quanto} & magazziniere \\
    \textbf{vorrei} & aggiungere automaticamente le informazioni del prodotto al sistema di tracciabilità quando lo pallettizzo per un ordine \\
    \textbf{così da} & evitare di doverlo fare manualmente \\
    \midrule
    \textbf{CoS} & Il sistema aggiunge automaticamente le informazioni del prodotto al sistema di tracciabilità quando viene scansionato il suo codice per aggiungerlo a un ordine \\
    \midrule
    \textbf{Priorità} & 5 \\
    \textbf{Story points} & 15 \\
    \bottomrule
  \end{tabularx}
  \label{user-story:i2-e1-u4}
\end{table}

\begin{table}[H]
  \begin{tabularx}{\textwidth}{lX}
    \toprule
    \textbf{I2-E2} & \textbf{Completare un ordine} \\
    \midrule
    \textbf{In quanto} & magazziniere \\
    \textbf{vorrei} & poter segnare un ordine come completo e pronto per essere spedito \\
    \textbf{così da} & poterlo spedire al cliente \\
    \midrule
    \textbf{CoS} & L'operatore può segnare un ordine come completo sfruttando un palmare \\
    \midrule
    \textbf{Priorità} & 3 \\
    \bottomrule
  \end{tabularx}
  \label{user-story:i2-e2}
\end{table}

\begin{table}[H]
  \begin{tabularx}{\textwidth}{lX}
    \toprule
    \textbf{I2-E2-U1} & \textbf{Segnare un ordine come completo} \\
    \midrule
    \textbf{In quanto} & magazziniere \\
    \textbf{vorrei} & avere la possibilità di segnare un ordine come completo \\
    \textbf{così da} & poterlo preparare per la spedizione al cliente \\
    \midrule
    \textbf{CoS} & È possibile interagire con il sistema per segnare un ordine come completo \\
    \midrule
    \textbf{Priorità} & 5 \\
    \textbf{Story points} & 7 \\
    \bottomrule
  \end{tabularx}
  \label{user-story:i2-e2-u1}
\end{table}

\begin{table}[H]
  \begin{tabularx}{\textwidth}{lX}
    \toprule
    \textbf{I2-E2-U2} & \textbf{Uso del palmare per segnare un ordine come completo} \\
    \midrule
    \textbf{In quanto} & magazziniere \\
    \textbf{vorrei} & utilizzare un palmare per segnare un ordine come completo \\
    \textbf{così da} & non dover inserire manualmente il codice dell'ordine per segnarlo come completo\\
    \midrule
    \textbf{CoS} & L'interfaccia del palmare permette di scansionare il codice dell'ordine per segnarlo come completo \\
    \midrule
    \textbf{Priorità} & 4 \\
    \textbf{Story points} & 6 \\
    \bottomrule
  \end{tabularx}
  \label{user-story:i2-e2-u2}
\end{table}

\begin{table}[H]
  \begin{tabularx}{\textwidth}{lX}
    \toprule
    \textbf{I2-E3} & \textbf{Generazione e stampa del DDT} \\
    \midrule
    \textbf{In quanto} & magazziniere \\
    \textbf{vorrei} & generare automaticamente un DDT per un ordine che è pronto per essere spedito \\
    \textbf{così da} & non doverlo fare manualmente \\
    \midrule
    \textbf{CoS} & Il sistema raccoglie automaticamente informazioni sui prodotti contenuti nel pallet dell'ordine e permette di stampare il DDT scansionando l'ID del pallet con un tablet \\
    \midrule
    \textbf{Priorità} & 2 \\
    \bottomrule
  \end{tabularx}
  \label{user-story:i2-e3}
\end{table}

\begin{table}[H]
  \begin{tabularx}{\textwidth}{lX}
    \toprule
    \textbf{I2-E3-U1} & \textbf{Generazione automatica del DDT} \\
    \midrule
    \textbf{In quanto} & magazziniere \\
    \textbf{vorrei} & generare automaticamente un DDT per un ordine pronto per la spedizione \\
    \textbf{così da} & evitare il processo manuale di compilazione del DDT \\
    \midrule
    \textbf{CoS} & Il sistema raccoglie automaticamente informazioni sui prodotti contenuti nell'ordine e permette di generare il DDT automaticamente \\
    \midrule
    \textbf{Priorità} & 5 \\
    \textbf{Story points} & 8 \\
    \bottomrule
  \end{tabularx}
  \label{user-story:i2-e3-u1}
\end{table}

\begin{table}[H]
  \begin{tabularx}{\textwidth}{lX}
    \toprule
    \textbf{I2-E3-U2} & \textbf{Utilizzo di un tablet per stampare il DDT} \\
    \midrule
    \textbf{In quanto} & magazziniere \\
    \textbf{vorrei} & utilizzare un tablet per stampare il DDT generato automaticamente per un ordine \\
    \textbf{così da} & non dover inserire manualmente il codice dell'ordine per stamparne il DDT \\
    \midrule
    \textbf{CoS} & L'interfaccia del tablet permette di mandare in stampa il DDT generato automaticamente scansionando l'ID del pallet dell'ordine \\
    \midrule
    \textbf{Priorità} & 3 \\
    \textbf{Story points} & 6 \\
    \bottomrule
  \end{tabularx}
  \label{user-story:i2-e3-u2}
\end{table}

\begin{table}[H]
  \begin{tabularx}{\textwidth}{lX}
    \toprule
    \textbf{I2-E4} & \textbf{Piazzamento ordini online} \\
    \midrule
    \textbf{In quanto} & cliente \\
    \textbf{vorrei} & poter sfruttare un portale di e-commerce per poter fare i miei acquisti \\
    \textbf{così da} & scegliere ed acquistare i prodotti che mi servono \\
    \midrule
    \textbf{CoS} & C'è un sito web che permette di scegliere quali prodotti acquistare tramite un catalogo e offre la possibilità di scegliere diverse metodologie di pagamento \\
    \midrule
    \textbf{Priorità} & 5 \\
    \bottomrule
  \end{tabularx}
  \label{user-story:i2-e4}
\end{table}

\begin{table}[H]
  \begin{tabularx}{\textwidth}{lX}
    \toprule
    \textbf{I2-E4-U1} & \textbf{Creazione sito web di Mambelli} \\
    \midrule
    \textbf{In quanto} & cliente \\
    \textbf{vorrei} & poter consultare il sito web del caseificio \\
    \textbf{così da} & avere un unico portale da consultare al bisogno \\
    \midrule
    \textbf{CoS} & C'è un sito web minimale del caseificio \\ 
    \midrule
    \textbf{Priorità} & 5 \\
    \textbf{Story points} & 20 \\
    \bottomrule
  \end{tabularx}
  \label{user-story:i2-e4-u1}
\end{table}

\begin{table}[H]
  \begin{tabularx}{\textwidth}{lX}
    \toprule
    \textbf{I2-E4-U2} & \textbf{Grafica del sito web} \\
    \midrule
    \textbf{In quanto} & Raffaella \\
    \textbf{vorrei} & che il nuovo sito mantenesse una grafica simile al precedente sito vetrina utilizzato dal caseificio \\
    \textbf{così da} & mantenere uno stile coerente riconoscibile dai clienti \\
    \midrule
    \textbf{CoS} & Il nuovo sito realizzato riprende il linguaggio visuale adottato dal precedente sito vetrina \\ 
    \midrule
    \textbf{Priorità} & 1 \\
    \textbf{Story points} & 7 \\
    \bottomrule
  \end{tabularx}
  \label{user-story:i2-e4-u2}
\end{table}

\begin{table}[H]
  \begin{tabularx}{\textwidth}{lX}
    \toprule
    \textbf{I2-E4-U3} & \textbf{Digitalizzazione del catalogo Mambelli} \\
    \midrule
    \textbf{In quanto} & Raffaella \\
    \textbf{vorrei} & una versione digitale dell'attuale catalogo di Mambelli \\
    \textbf{così da} & poter gestire con più facilità il catalogo dei prodotti del caseificio \\
    \midrule
    \textbf{CoS} & Il catalogo è salvato in una base di dati \\
    \midrule
    \textbf{Priorità} & 4 \\
    \textbf{Story points} & 2 \\
    \bottomrule
  \end{tabularx}
  \label{user-story:i2-e4-u3}
\end{table}

\begin{table}[H]
  \begin{tabularx}{\textwidth}{lX}
    \toprule
    \textbf{I2-E4-U4} & \textbf{Modifica del catalogo Mambelli} \\
    \midrule
    \textbf{In quanto} & Raffaella \\
    \textbf{vorrei} & poter modificare il catalogo dei prodotti \\
    \textbf{così da} & poter aggiornare le descrizioni dei prodotti del caseificio o aggiungerne di nuovi qualora necessario \\
    \midrule
    \textbf{CoS} & C'è una semplice interfaccia che permette a Raffaella di modificare il catalogo memorizzato \\
    \midrule
    \textbf{Priorità} & 2 \\
    \textbf{Story points} & 5 \\
    \bottomrule
  \end{tabularx}
  \label{user-story:i2-e4-u4}
\end{table}

\begin{table}[H]
  \begin{tabularx}{\textwidth}{lX}
    \toprule
    \textbf{I2-E4-U5} & \textbf{Visualizzazione del catalogo online} \\
    \midrule
    \textbf{In quanto} & cliente \\
    \textbf{vorrei} & sfogliare un catalogo online di tutti i prodotti disponibili \\
    \textbf{così da} & scegliere cosa comprare \\
    \midrule
    \textbf{CoS} & È possibile consultare nel sito web il catalogo di tutti i prodotti che possono essere acquistati presso il caseificio \\
    \midrule
    \textbf{Priorità} & 5 \\
    \textbf{Story points} & 3 \\
    \bottomrule
  \end{tabularx}
  \label{user-story:i2-e4-u5}
\end{table}

\begin{table}[H]
  \begin{tabularx}{\textwidth}{lX}
    \toprule
    \textbf{I2-E4-U6} & \textbf{Carrello per gli acquisti online} \\
    \midrule
    \textbf{In quanto} & cliente \\
    \textbf{vorrei} & mettere i formaggi a cui sono interessato in un carrello \\
    \textbf{così da} & poterli ordinare online \\
    \midrule
    \textbf{CoS} & Il sito web permette di mettere i prodotti in un carrello per poterli poi ordinare \\
    \midrule
    \textbf{Priorità} & 5 \\
    \textbf{Story points} & 7 \\
    \bottomrule
  \end{tabularx}
  \label{user-story:i2-e4-u6}
\end{table}

\begin{table}[H]
  \begin{tabularx}{\textwidth}{lX}
    \toprule
    \textbf{I2-E4-U7} & \textbf{Ordini online} \\
    \midrule
    \textbf{In quanto} & cliente \\
    \textbf{vorrei} & poter fare un ordine con tutti i miei prodotti messi nel carrello \\
    \textbf{così da} & comprare i formaggi di cui ho bisogno \\
    \midrule
    \textbf{CoS} & C'è un pulsante per effettuare l'acquisto dei prodotti che ho messo nel carrello \\
    \midrule
    \textbf{Priorità} & 5 \\
    \textbf{Story points} & 12 \\
    \bottomrule
  \end{tabularx}
  \label{user-story:i2-e4-u7}
\end{table}

\begin{table}[H]
  \begin{tabularx}{\textwidth}{lX}
    \toprule
    \textbf{I2-E4-U8} & \textbf{Metodi di pagamento} \\
    \midrule
    \textbf{In quanto} & cliente \\
    \textbf{vorrei} & poter scegliere fra più metodologie di pagamento \\
    \textbf{così da} & scegliere quella che mi conviene di più \\
    \midrule
    \textbf{CoS} & Un cliente può scegliere di pagare con un bonifico, in contanti alla consegna oppure tramite carta di credito \\
    \midrule
    \textbf{Priorità} & 5 \\
    \textbf{Story points} & 15 \\
    \bottomrule
  \end{tabularx}
  \label{user-story:i2-e4-u8}
\end{table}

\begin{table}[H]
  \begin{tabularx}{\textwidth}{lX}
    \toprule
    \textbf{I2-E4-U9} & \textbf{Unificazione dell'e-commerce e del sito vetrina} \\
    \midrule
    \textbf{In quanto} & cliente \\
    \textbf{vorrei} & un unico sito con informazioni sul caseificio e l'e-commerce \\
    \textbf{così da} & non dover cambiare sito per piazzare i miei ordini una volta aver consultato la vetrina \\
    \midrule
    \textbf{CoS} & Il sito web realizzato per l'e-commerce contiene anche varie informazioni sul caseificio \\ 
    \midrule
    \textbf{Priorità} & 3 \\
    \textbf{Story points} & 5 \\
    \bottomrule
  \end{tabularx}
  \label{user-story:i2-e4-u9}
\end{table}

\begin{table}[H]
  \begin{tabularx}{\textwidth}{lX}
    \toprule
    \textbf{I2-E4-U10} & \textbf{Sito web sempre disponibile} \\
    \midrule
    \textbf{In quanto} & Raffaella \\
    \textbf{vorrei} & che il sito fosse sempre raggiungibile senza disservizi \\
    \textbf{così da} & non spingere i clienti verso la concorrenza se dovessero trovare il sito offline \\
    \midrule
    \textbf{CoS} & L'infrastruttura è sufficiente a gestire il flusso di clienti del caseificio \\
    & Il sito web è disponibile nel 99\% del tempo in un anno \\
    \midrule
    \textbf{Priorità} & 4 \\
    \textbf{Story points} & 10 \\
    \bottomrule
  \end{tabularx}
  \label{user-story:i2-e4-u10}
\end{table}

\begin{table}[H]
  \begin{tabularx}{\textwidth}{lX}
    \toprule
    \textbf{I2-E4-U11} & \textbf{User experience del sito} \\
    \midrule
    \textbf{In quanto} & cliente \\
    \textbf{vorrei} & che il sito fosse facile da utilizzare \\
    \textbf{così da} & poter effettuare gli ordini senza difficoltà \\
    \midrule
    \textbf{CoS} & Il sito rispetta le Euristiche di Nielsen \\
    & Almeno il 90\% degli utenti finali che hanno testato i mockup del sito lo hanno valutato positivamente \\
    \midrule
    \textbf{Priorità} & 3 \\
    \textbf{Story points} & 13 \\
    \bottomrule
  \end{tabularx}
  \label{user-story:i2-e4-u11}
\end{table}

\section*{I3: Gestione del magazzino}

\begin{table}[H]
  \begin{tabularx}{\textwidth}{lX}
    \toprule
    \textbf{I3-U1} & \textbf{Tracciamento automatico della disponibilità dei prodotti} \\
    \midrule
    \textbf{In quanto} & magazziniere \\
    \textbf{vorrei} & che il sistema tenga traccia automaticamente della disponibilità dei prodotti \\
    \textbf{così da} & non doverlo fare manualmente \\
    \midrule
    \textbf{CoS} & Il sistema tiene automaticamente traccia delle disponibilità di ciascun prodotto e le aggiorna mano a mano che i prodotti vengono tolti dal magazzino per essere spediti ai clienti \\
    \midrule
    \textbf{Priorità} & 3 \\
    \textbf{Story points} & 5 \\
    \bottomrule
  \end{tabularx}
  \label{user-story:i3-u1}
\end{table}


\begin{table}[H]
  \begin{tabularx}{\textwidth}{lX}
    \toprule
    \textbf{I3-U2} & \textbf{Visualizzazione della disponibilità dei prodotti} \\
    \midrule
    \textbf{In quanto} & magazziniere \\
    \textbf{vorrei} & avere la possibilità di visualizzare la disponibilità di ciascun prodotto \\
    \textbf{così da} & sapere quali prodotti sono disponibili \\
    \midrule
    \textbf{CoS} & L'interfaccia permette di visualizzare la disponibilità di ciascun prodotto in tempo reale \\
    \midrule
    \textbf{Priorità} & 3 \\
    \textbf{Story points} & 3 \\
    \bottomrule
  \end{tabularx}
  \label{user-story:i3-u2}
\end{table}
