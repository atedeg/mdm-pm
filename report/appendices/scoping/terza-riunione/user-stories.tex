\chapter{User Stories}
\label{app:user-stories}

\section{Pianificazione della produzione}

\begin{tabularx}{\textwidth}{lX}
\toprule
\multicolumn{2}{l}{\textbf{Generazione del piano di produzione}} \\
\midrule
\textbf{In quanto} & Raffaella \\
\textbf{vorrei} & generare automaticamente un piano di produzione per il giorno \\
\textbf{così da} & ottimizzare l'utilizzo delle macchine per la produzione di formaggio durante la giornata \\
\midrule
\textbf{CoS} & Il piano di produzione è adeguato per soddisfare tutti gli ordini in tempo \\
& Le macchine sono inattive per al massimo il 15\% del tempo \\
\bottomrule
\end{tabularx}

\begin{tabularx}{\textwidth}{lX}
  \toprule
  \multicolumn{2}{l}{\textbf{Visualizzazione del piano di produzione}} \\
  \midrule
  \textbf{In quanto} & Raffaella \\
  \textbf{vorrei} & visualizzare il piano di produzione attuale \\
  \textbf{così da} & conoscere ciò che è stato generato \\
  \midrule
  \textbf{CoS} & L'interfaccia è facile da usare e comprendere secondo le Euristiche di Nielsen~\cite{cit:nielsen} \\
  \bottomrule
\end{tabularx}

\begin{tabularx}{\textwidth}{lX}
  \toprule
  \multicolumn{2}{l}{\textbf{Modifica del piano di produzione}} \\
  \midrule
  \textbf{In quanto} & Raffaella \\
  \textbf{vorrei} & modificare il piano di produzione generato per il giorno \\
  \textbf{così da} & tenere conto di situazioni eccezionali che potrebbero verificarsi durante la giornata \\
  \midrule
  \textbf{CoS} & Il piano di produzione può essere modificato prima di essere approvato per la produzione \\
  \bottomrule
\end{tabularx}

\section{Ordini dei clienti}

\begin{tabularx}{\textwidth}{lX}
  \toprule
  \multicolumn{2}{l}{\textbf{Prezzatura automatica}} \\
  \midrule
  \textbf{In quanto} & magazziniere \\
  \textbf{vorrei} & che il sistema calcolasse automaticamente i prezzi degli ordini in entrata \\
  \textbf{così da} & evitare di doverlo fare manualmente sfogliando l'elenco dei prezzi \\
  \midrule
  \textbf{CoS} & Il sistema calcola automaticamente i prezzi degli ordini in entrata utilizzando una versione digitale del listino prezzi attualmente in uso \\
  \bottomrule
\end{tabularx}

\begin{tabularx}{\textwidth}{lX}
  \toprule
  \multicolumn{2}{l}{\textbf{Pallettizzazione automatica}} \\
  \midrule
  \textbf{In quanto} & magazziniere \\
  \textbf{vorrei} & pallettizzare un prodotto per un ordine \\
  \textbf{così da} & poter mettere da parte un prodotto per l'ordine di un cliente \\
  \midrule
  \textbf{CoS} & Il magazziniere può scansionare il codice a barre del prodotto con un palmare per contrassegnarlo come pallettizzato e aggiornare le forniture del magazzino \\
  \bottomrule
\end{tabularx}

\begin{tabularx}{\textwidth}{lX}
  \toprule
  \multicolumn{2}{l}{\textbf{Integrazione con il sistema di tracciabilità}} \\
  \midrule
  \textbf{In quanto} & magazziniere \\
  \textbf{vorrei} & aggiungere automaticamente le informazioni del prodotto al sistema di tracciabilità quando lo pallettizzo per un ordine \\
  \textbf{così da} & evitare di doverlo fare manualmente \\
  \midrule
  \textbf{CoS} & Il sistema aggiunge automaticamente le informazioni del prodotto al sistema di tracciabilità quando viene scansionato il suo codice per aggiungerlo a un ordine \\
  \bottomrule
\end{tabularx}

\begin{tabularx}{\textwidth}{lX}
  \toprule
  \multicolumn{2}{l}{\textbf{Completare un ordine}} \\
  \midrule
  \textbf{In quanto} & magazziniere \\
  \textbf{vorrei} & poter segnare un ordine come completo e pronto per essere spedito \\
  \textbf{così da} & poterlo spedire al cliente \\
  \midrule
  \textbf{CoS} & L'operatore può segnare un ordine come completo sfruttando un palmare \\
  \bottomrule
\end{tabularx}

\begin{tabularx}{\textwidth}{lX}
  \toprule
  \multicolumn{2}{l}{\textbf{Generazione automatica del DDT}} \\
  \midrule
  \textbf{In quanto} & magazziniere \\
  \textbf{vorrei} & generare automaticamente un DDT per un ordine che è pronto per essere spedito \\
  \textbf{così da} & non doverlo fare manualmente \\
  \midrule
  \textbf{CoS} & Il sistema raccoglie automaticamente informazioni sui prodotti contenuti nel pallet dell'ordine e permette di stampare il DDT scansionando l'ID del pallet con un tablet \\
  \bottomrule
\end{tabularx}

\subsection{Epic: Online Order Placing}
\begin{tabularx}{\textwidth}{lX}
  \toprule
  \multicolumn{2}{l}{\textbf{Catalogo online}} \\
  \midrule
  \textbf{In quanto} & cliente \\
  \textbf{vorrei} & sfogliare un catalogo online di tutti i prodotti disponibili \\
  \textbf{così da} & scegliere cosa comprare \\
  \midrule
  \textbf{CoS} & C'è un sito web con un catalogo di tutti i prodotti disponibili \\
  \bottomrule
\end{tabularx}

\begin{tabularx}{\textwidth}{lX}
  \toprule
  \multicolumn{2}{l}{\textbf{Carrello per gli acquisti online}} \\
  \midrule
  \textbf{In quanto} & cliente \\
  \textbf{vorrei} & mettere i formaggi a cui sono interessato in un carrello \\
  \textbf{così da} & poterli ordinare online \\
  \midrule
  \textbf{CoS} & Il sito web permette di mettere i prodotti in un carrello per poterli poi ordinare \\
  \bottomrule
\end{tabularx}

\begin{tabularx}{\textwidth}{lX}
  \toprule
  \multicolumn{2}{l}{\textbf{Ordini online}} \\
  \midrule
  \textbf{In quanto} & cliente \\
  \textbf{vorrei} & poter fare un ordine con tutti i miei prodotti messi nel carrello \\
  \textbf{così da} & comprare i formaggi di cui ho bisogno \\
  \midrule
  \textbf{CoS} & C'è un pulsante per effettuare l'acquisto dei prodotti che ho messo nel carrello \\
  \bottomrule
\end{tabularx}

\begin{tabularx}{\textwidth}{lX}
  \toprule
  \multicolumn{2}{l}{\textbf{Metodi di pagamento}} \\
  \midrule
  \textbf{In quanto} & cliente \\
  \textbf{vorrei} & poter scegliere fra più metodologie di pagamento \\
  \textbf{così da} & scegliere quella che mi conviene di più \\
  \midrule
  \textbf{CoS} & Un cliente può pagare con un bonifico \\
  & Un cliente può pagare in contanti alla consegna \\
  & Un cliente può pagare con una carta di credito \\
  \bottomrule
\end{tabularx}

\begin{tabularx}{\textwidth}{lX}
  \toprule
  \multicolumn{2}{l}{\textbf{User experience del sito}} \\
  \midrule
  \textbf{In quanto} & cliente \\
  \textbf{vorrei} & che il sito fosse facile da utilizzare \\
  \textbf{così da} & poter effettuare gli ordini senza difficoltà \\
  \midrule
  \textbf{CoS} & Il sito rispetta le Euristiche di Nielsen \\
  & Almeno il 90\% degli utenti finali che hanno testato i mockup del sito lo hanno valutato positivamente \\
  \bottomrule
\end{tabularx}

\section{Magazzino}
\begin{tabularx}{\textwidth}{lX}
  \toprule
  \multicolumn{2}{l}{\textbf{Tracciamento delle disponibilità dei prodotti}} \\
  \midrule
  \textbf{In quanto} & magazziniere \\
  \textbf{vorrei} & vedere la disponibilità di ciascun prodotto in tempo reale \\
  \textbf{così da} & sapere quali prodotti sono disponibili \\
  \midrule
  \textbf{CoS} & Il sistema tiene automaticamente traccia delle disponibilità di ciascun prodotto e le aggiorna mano a mano che i prodotti vengono tolti dal magazzino per essere spediti ai clienti \\
  \bottomrule
\end{tabularx}
