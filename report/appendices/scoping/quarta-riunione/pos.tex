\chapter{Project Overview Statement}
\label{app:pos}

\section{Problems/Opportunities}
L'azienda ha riscontrato un grande aumento nel volume degli ordini, rendendo difficile gestirli con un processo manuale come sempre fatto fino ad ora.
Per questo motivo ha bisogno di un sistema che permetta di raccogliere e gestire in maniera efficiente gli ordini dei clienti; questo sistema deve essere realizzato categoricamente entro il 30 settembre per poter essere sfruttato nella gestione del flusso di ordini del periodo natalizio. Inoltre, l'azienda necessita di un sistema che permetta di automatizzare la pianificazione della produzione dei prodotti per aumentare l'efficienza nell'uso dei macchinari. Infatti, con una pianificazione manuale si è osservato come ci siano fasce orarie in cui i macchinari non sono utilizzati.
Infine, è importante che il nuovo sistema di gestione degli ordini si integri automaticamente con il sistema della tracciabilità adottato dall'azienda.

\section{Goals}
\begin{itemize}
  \item Unificazione dei canali per l'acquisto dei prodotti da parte dei clienti entro il 30 settembre
  \item Centralizzazione della gestione degli ordini e della loro tracciabilità
  \item Automazione della pianificazione della produzione per massimizzare la produttività dei macchinari
\end{itemize}

\section{Objectives}
\begin{itemize}
  \item Realizzazione di un portale web per permettere ai clienti di piazzare ordini
  \item Realizzazione di un sistema che permetta di gestire automaticamente il ciclo di vita degli ordini
  \item Integrazione del sistema degli ordini con il sistema della tracciabilità preesistente
  \item Realizzazione di un sistema che permetta di pianificare automaticamente la produzione giornaliera
\end{itemize}

\section{Success Criteria}
\begin{itemize}
  \item Il portale web viene utilizzato per effettuare almeno il 90\% degli ordini
  \item Il sistema di pianificazione della produzione deve realizzare un piano entro dieci minuti dal momento della richiesta
  \item Il piano generato automaticamente deve permettere di soddisfare il 95\% degli ordini ricevuti
  \item Il piano generato automaticamente deve far in modo che i macchinari siano sempre utilizzati al massimo della loro capacità
  \item Non è necessario alcun intervento manuale aggiuntivo per gestire la tracciabilità dei prodotti negli ordini dei clienti
\end{itemize}

\section{Assumptions}
\begin{itemize}
  \item Il cliente può sostenere i costi dovuti alla realizzazione dell'intero progetto
  \item Il cliente dispone di dati storici sulla pianificazione della produzione
  \item I palmari a disposizione dei magazzinieri sono sufficientemente moderni per potersi connettere a internet
\end{itemize}

\section{Risks}
\begin{itemize}
  \item I clienti abituati a contattare direttamente l'azienda per effettuare gli ordini potrebbero non adeguarsi al nuovo sistema
  \item La normativa europea sui documenti di trasporto potrebbe cambiare durante la realizzazione del progetto
  \item L'integrazione con il sistema della tracciabilità potrebbe non essere possibile o molto laboriosa
  \item Il cliente potrebbe aggiungere una nuova linea di prodotti durante lo sviluppo del sistema
  \item Potrebbe non essere possibile effettuare la consegna del sito di e-commerce con la gestione degli ordini entro il 30 settembre
  \item I dati storici relativi alla pianificazione della produzione potrebbero essere di qualità insufficiente per realizzare in maniera efficace un sistema di pianificazione automatica
  \item Potrebbero non essere disponibili utenti finali del sistema per effettuare le prove di usabilità tramite focus group
\end{itemize}

\section{Obstacles}
\begin{itemize}
  \item Il sistema di gestione degli ordini deve essere pronto entro il 30 settembre per poter essere utilizzato per la stagione invernale
  \item Alcuni membri del team non hanno esperienza di DDD e richiedono lezioni di formazione
\end{itemize}

