\chapter{Ubiquitous Language}
\label{app:ubiquitous-language}

\section{Definizioni condivise}
\begin{description}
    \item[Tipo di formaggio] La tipologia di un formaggio prodotto.
    \item[Prodotto] Un tipo di formaggio con il suo rispettivo peso.
    \item[Ingrediente] Un ingrediente necessario per produrre un tipo di formaggio.
\end{description}

\section{Pianificazione del riordino del latte}
\begin{description}
    \item [Latte processato] I quintali di latte processati per poter produrre dei formaggi.
    \item [Quintali di latte] Una quantità di latte espressa in quintali.
    \item [Resa] Un decimale che rappresenta la resa del latte nella produzione di un determinato tipo di formaggio: \textit{i.e.} per produrre $n$ quintali di un certo formaggio con una resa $r$ servono $r \cdot n$ quintali di latte.
    \item [Ricettario] Definisce, per ogni tipologia di formaggio, la resa del latte nella sua produzione.
    \item [Magazzino] Definisce, per ogni prodotto, la quantità disponibile.
    \item [Quantità in magazzino] La quantità di un prodotto in magazzino, potrebbe anche essere 0.
    \item [Quantità] La quantità di qualcosa.
    \item [Prodotto richiesto] Un prodotto richiesto in una certa quantità che deve essere pronto entro una certa data.
\end{description}

\section{Pianificazione della produzione}
\begin{description}
    \item [Piano di produzione] Tutti i prodotti che devono essere prodotti in un giorno.
    \item [Prodotto da produrre] Un formaggio che deve essere prodotto in una data quantità.
    \item [Prodotto mancante] Un prodotto con il numero di unità mancanti che devono essere aggiunte al magazzino.
    \item [Ordine] Un insieme di prodotti ordinati e una data entro la quale l'ordine deve essere completato.
    \item [Prodotto ordinato] Un prodotto richiesto in una certa quantità.
    \item [Giorni di stagionatura] i giorni di invecchiamento in cella che devono passare per un formaggio prima che questo sia pronto.
\end{description}

\section{Produzione}
\begin{description}
    \item [Piano di produzione] Si compone di linee che specificano cosa produrre.
    \item [Line del piano di produzione] Una linea di un piano di produzione che specifica un prodotto e quante unità di esso produrre.
    \item [Produzione da avviare] Una produzione che deve essere avviata, ha un ID che ne identifica il lotto e specifica il prodotto e le unità realizzate.
    \item [Produzione avviata] Una produzione che è stata avviata, ha un ID che ne identifica il lotto e specifica il prodotto e le unità realizzate.
    \item [Produzione conclusa] Una produzione che è stata conclusa, ha un ID che ne identifica il lotto e specifica il prodotto e le unità realizzate.
    \item [Ricettario] Associa ad ogni tipo di formaggio la ricetta per produrlo.
    \item [Ricetta] Una lista di ingredienti e i rispettivi quintali necessari per produrre un quintale di un prodotto.
    \item [Quintali di ingrediente] Un ingrediente e un peso in quintali.
    \item [Giorni di stagionatura] I giorni di invecchiamento in cella che devono passare per un formaggio prima che questo sia pronto.
\end{description}

\section{Magazzino}
\begin{description}
    \item [Stock disponibile] I prodotti attualmente disponibili in magazzino; ogni prodotto è disponibile in una certa quantità (che può anche essere zero se il prodotto è esaurito).
    \item [Stock desiderato] La quantità desiderata di ogni prodotto che deve essere sempre presente in magazzino per avere un certo margine di sicurezza per la gestione delle consegne.
    \item [Quantità disponibile] La quantità di un prodotto disponibile in magazzino.
    \item [Quantità desiderata] La quantità desiderata di un prodotto che dovrebbe essere presente in magazzino.
    \item [Quantità mancante] La quantità di un prodotto mancante per raggiungere il livello di stock desiderato.
    \item [Lotto] Un lotto di prodotti di un certo tipo, identificato da un ID, che non è stato ancora sottoposto a controllo di qualità.
    \item [Lotto certificato] Un lotto di prodotti di un certo tipo, identificato da un ID, che è stato sottoposto a controllo di qualità.
    \item [Prodotto etichettato] Un prodotto con la sua quantità e l'ID del lotto a cui appartiene.
\end{description}

\section{Rifornimento}
\begin{description}
    \item[Magazzino] Definisce per ogni ingrediente la quantità disponibile.
    \item[Latte in magazzino] Quintali di latte in magazzino.
    \item[Quintali di ingrediente] Un ingrediente e un peso in quintali.
    \item[Quintali di latte] Una quantità di latte espressa in quintali.
\end{description}

\section{Ordini dei clienti}
\begin{description}
    \item[Ordine in arrivo]
    Un insieme di linee d'ordine con le rispettive quantità (ad esempio 1000 ricotte di 0,5 kg, 50 squacqueroni di 1 kg), contiene anche dati sul cliente, una data di consegna prevista e il luogo di consegna.
    \item[Linea di un ordine in arrivo] Un prodotto con la qua quantità ordinata.
    \item[Cliente] Un'entità fisica o giuridica che piazza gli ordini.
    \item[Luogo] Il luogo dove un ordine deve essere spedito.
    \item[Ordine prezzato] Un ordine dove ogni linea ha un prezzo associato e, opzionalmente, uno sconto applicato. Ha anche un prezzo totale. La sua struttura è simile a quella dell'ordine in arrivo con la differenza che ogni linea è stata prezzata.
    \item[Line di un ordine prezzato] Un prodotto con la sua quantità e un prezzo.
    \item[Ordine in elaborazione] Un ordine che sta venendo elaborato da un operatore. La sua struttura è simile a quella dell'ordine prezzato con la differenza che ogni linea può specificare se è stata completata o meno.
    \item[Linea di un ordine in elaborazione] Un prodotto con la sua quantità, un prezzo e lo stato di completamento. Può essere in due stati: completa se il prodotto è già stato palletizzato e è pronto nella quantità richiesta; incompleta se il prodotto non è presente nella quantità richiesta.
    \item[Ordine completato] Un ordine che è stato completato da un operatore ed è pronto per essere spedito.
    \item[Linea di un ordine completato] Un prodotto con la sua quantità e un prezzo.
    \item[DDT] Un documento che deve specificare: il luogo di consegna, il luogo di spedizione, i dati del cliente, la data di spedizione, il peso totale del pallet e un insieme di linee del DDT.
    \item[Linea del DDT] Un prodotto con la sua quantità spedita.
\end{description}

\section{Prezzatura}
\begin{description}
    \item[Linea d'ordine in arrivo] Un prodotto con la sua quantità ordinata.
    \item[Quantità] Una quantità di qualcosa.
    \item[Listino] Associa a ogni prodotto il suo prezzo unitario.
    \item[Prezzo] Un prezzo espresso in centesimi d'euro.
    \item[Client] Un'entità fisica o giuridica che piazza gli ordini.
    \item[ID Cliente] Un ID che identifica in maniera univoca un cliente.
    \item[Promozione] Una promozione per un cliente, con una data di scadenza, che contiene delle linee di promozione per alcuni prodotti.
    \item[Linea di promozione fissa] Questa linea di promozione specifica il prodotto scontato e di quanto scontarlo. Ogni linea d'ordine che contiene il prodotto è scontata di quanto specificato.
    \item[Linea di promozione a soglia] Questa linea di promozione specifica il prodotto scontato, la soglia e di quanto scontarlo. Solo ordinati in quantità superiore alla soglia sono scontati; gli altri sono a prezzo pieno.
    \item[Soglia] Una quantità che indica il numero minimo di prodotti da ordinare perché uno sconto possa essere applicato.
    \item[Percentuale di sconto] Una percentuale di sconto espressa con un numero compreso fra 0 (escluso) e 100 (incluso).
\end{description}