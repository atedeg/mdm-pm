\chapter{Analisi dei rischi}
\label{app:analisi-rischi}

\begin{table}[H]
  \begin{tabularx}{\textwidth}{lX}
    \toprule
    \textbf{Rischio}            & \textbf{I clienti abituati a contattare direttamente l'azienda per effettuare gli ordini potrebbero non adeguarsi al nuovo sistema}                                                \\
    \midrule
    \textbf{Probabilità}        & Probabile                                                                                                                                                                          \\
    \textbf{Impatto}            & Critico                                                                                                                                                                            \\
    \textbf{Livello di rischio} & Alto                                                                                                                                                                               \\
    \textbf{Mitigazione}        & Grande attenzione alla user experience del prodotto realizzato sfruttando anche focus group in modo da assicurarsi che i clienti possano adattarsi facilmente alla nuova soluzione \\
                                & Campagna di promozione per incentivare l'adozione del nuovo sistema per esempio con degli sconti                                                                                   \\
    \bottomrule
  \end{tabularx}
\end{table}

\begin{table}[H]
  \begin{tabularx}{\textwidth}{lX}
    \toprule
    \textbf{Rischio}            & \textbf{La normativa europea sui documenti di trasporto potrebbe cambiare durante la realizzazione del progetto}                      \\
    \midrule
    \textbf{Probabilità}        & Possibile                                                                                                                             \\
    \textbf{Impatto}            & Critico                                                                                                                               \\
    \textbf{Livello di rischio} & Alto                                                                                                                                  \\
    \textbf{Mitigazione}        & Analizzare la proposta di legge per individuare quali porzioni del sistema sarebbero interessate in modo da anticipare il cambiamento \\
    \bottomrule
  \end{tabularx}
\end{table}

\begin{table}[H]
  \begin{tabularx}{\textwidth}{lX}
    \toprule
    \textbf{Rischio}            & \textbf{L'integrazione con il sistema della tracciabilità potrebbe non essere possibile}                                                                      \\
    \midrule
    \textbf{Probabilità}        & Improbabile                                                                                                                                                   \\
    \textbf{Impatto}            & Catastrofico                                                                                                                                                  \\
    \textbf{Livello di rischio} & Alto                                                                                                                                                          \\
    \textbf{Mitigazione}        & Verificare prima di accettare il progetto che sia effettivamente possibile interagire (tramite un'API o altre tecnologie) con il servizio della tracciabilità \\
    \bottomrule
  \end{tabularx}
\end{table}

\begin{table}[H]
  \begin{tabularx}{\textwidth}{lX}
    \toprule
    \textbf{Rischio}            & \textbf{L'integrazione con il sistema della tracciabilità potrebbe richiedere più tempo del previsto} \\
    \midrule
    \textbf{Probabilità}        & Possibile                                                                                             \\
    \textbf{Impatto}            & Marginale                                                                                             \\
    \textbf{Livello di rischio} & Medio                                                                                                 \\
    \textbf{Mitigazione}        & Dedicare più risorse all'integrazione durante lo sprint in cui verrà svolto                           \\
    \bottomrule
  \end{tabularx}
\end{table}

\begin{table}[H]
  \begin{tabularx}{\textwidth}{lX}
    \toprule
    \textbf{Rischio}            & \textbf{Il cliente potrebbe aggiungere una nuova linea di prodotti durante lo sviluppo del sistema}     \\
    \midrule
    \textbf{Probabilità}        & Possibile                                                                                               \\
    \textbf{Impatto}            & Marginale                                                                                               \\
    \textbf{Livello di rischio} & Medio                                                                                                   \\
    \textbf{Mitigazione}        & Rendere il sistema flessibile in modo da poter supportare l'aggiunta di nuovi prodotti in corso d'opera \\
    \bottomrule
  \end{tabularx}
\end{table}

\begin{table}[H]
  \begin{tabularx}{\textwidth}{lX}
    \toprule
    \textbf{Rischio}            & \textbf{Potrebbe non essere possibile effettuare la consegna del sito di e-commerce con la gestione degli ordini entro il 30 settembre} \\
    \midrule
    \textbf{Probabilità}        & Improbabile                                                                                                                             \\
    \textbf{Impatto}            & Catastrofico                                                                                                                            \\
    \textbf{Livello di rischio} & Alto                                                                                                                                    \\
    \textbf{Mitigazione}        & Dare massima priorità alla realizzazione dell'e-commerce                                                                                \\
                                & Realizzare subito una prima interfaccia minimale da far validare al cliente                                                             \\
                                & Risparmiare tempo riutilizzando componenti già sviluppati in precedenza per altri progetti                                              \\
    \bottomrule
  \end{tabularx}
\end{table}

\begin{table}[H]
  \begin{tabularx}{\textwidth}{lX}
    \toprule
    \textbf{Rischio}            & \textbf{I dati storici relativi alla pianificazione della produzione potrebbero essere di qualità insufficiente per realizzare in maniera efficace un sistema di pianificazione automatica} \\
    \midrule
    \textbf{Probabilità}        & Possibile                                                                                                                                                                                   \\
    \textbf{Impatto}            & Critico                                                                                                                                                                                     \\
    \textbf{Livello di rischio} & Alto                                                                                                                                                                                        \\
    \textbf{Mitigazione}        & Prima di accettare il progetto verificare la presenza di dati per la pianificazione automatica                                                                                              \\
                                & Se non presenti, si potrebbe rimandare la pianificazione automatica a un secondo momento e avviare un processo di raccolta dei dati storici sulla pianificazione nel formato più adatto     \\
    \bottomrule
  \end{tabularx}
\end{table}

\begin{table}[H]
  \begin{tabularx}{\textwidth}{lX}
    \toprule
    \textbf{Rischio}            & \textbf{Nicolas prenderà parte alla conferenza \href{https://zfoh.ch/zurihac2022/}{ZuriHack} dal 11 giugno al 13 giugno} \\
    \midrule
    \textbf{Probabilità}        & Certo                                                                                                                    \\
    \textbf{Impatto}            & Minore                                                                                                                   \\
    \textbf{Livello di rischio} & Alto                                                                                                                     \\
    \textbf{Mitigazione}        & Ridurre il numero di user story nello sprint interessato                                                                 \\
    \bottomrule
  \end{tabularx}
\end{table}

\begin{table}[H]
  \begin{tabularx}{\textwidth}{lX}
    \toprule
    \textbf{Rischio}            & \textbf{Sviluppatori potrebbero essere irreperibili per via di situazioni eccezionali (malattia, infortuni, ecc.)}                                                                                                                       \\
    \midrule
    \textbf{Probabilità}        & Possibile                                                                                                                                                                                                                                \\
    \textbf{Impatto}            & Critico                                                                                                                                                                                                                                  \\
    \textbf{Livello di rischio} & Alto                                                                                                                                                                                                                                     \\
    \textbf{Mitigazione}        & In caso di assenze prolungate si può considerare l'assunzione di nuovo personale a progetto                                                                                                                                              \\
                                & La schedula degli sprint viene fatta in modo da non occupare tutti gli sviluppatori al 100\%; viene lasciato un certo slack in modo che sia possibile per uno sviluppatore coprire -- anche solo parzialmente -- l'assenza di un collega \\
                                & Rimodulazione dello sprint backlog per adattare il carico di lavoro agli sviluppatori rimanenti                                                                                                                                          \\
    \bottomrule
  \end{tabularx}
\end{table}

\begin{table}[H]
  \begin{tabularx}{\textwidth}{lX}
    \toprule
    \textbf{Rischio}            & \textbf{Potrebbero non essere disponibili utenti finali del sistema per effettuare le prove di usabilità tramite focus group}                                              \\
    \midrule
    \textbf{Probabilità}        & Improbabile                                                                                                                                                                \\
    \textbf{Impatto}            & Marginale                                                                                                                                                                  \\
    \textbf{Livello di rischio} & Medio                                                                                                                                                                      \\
    \textbf{Mitigazione}        & Fin da subito iniziare a prendere contatti con i possibili clienti in modo da ricevere il prima possibile le loro disponibilità e organizzare i focus group in tempi utili \\
                                & Fornire un incentivo economico alla partecipazione ai focus group (per esempio sotto forma di sconto sulle merci ordinate)                                                 \\
                                & Utilizzo di componenti grafiche standard e di qualità per avere una buona baseline di usabilità                                                                            \\
    \bottomrule
  \end{tabularx}
\end{table}
