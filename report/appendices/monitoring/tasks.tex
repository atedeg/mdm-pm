\chapter{Sprint backlog del terzo sprint}
\label{app:sprint-backlog}

In seguito è riportato un elenco che mostra le storie pianificate per il terzo sprint con la relativa scomposizione in task e loro stime

\begin{itemize}
  \item \hyperref[user-story:i2-e4-u2]{I2-E4-U2}: Grafica del sito web
        \begin{itemize}
          \item Ottenere i vecchi fogli di stile adottati dal caseificio (effort 1)
          \item Adattare il vecchio stile al nuovo sito (effort 4)
        \end{itemize}
  \item \hyperref[user-story:i2-e4-u4]{I2-E4-U4}: Modifica del catalogo Mambelli
        \begin{itemize}
          \item Definire lo schema del DB (effort 4)
          \item Definire interfaccia per comunicare con il database (effort 3)
          \item Implementare query per aggiornare il database (effort 2)
          \item Realizzare form per poter permettere all'utente di modificare il database (effort 2)
        \end{itemize}
  \item \hyperref[user-story:i2-e4-u9]{I2-E4-U9}: Unificazione dell'e-commerce e del sito vetrina
        \begin{itemize}
          \item Realizzare un mockup del risultato finale (effort 11)
          \item Richiedere la validazione del cliente del mockup (effort 3)
          \item Integrare il mockup nel sito effettivo (effort 7)
          \item Integrare gli stili adottati per il sito effettivo (effort 5)
        \end{itemize}
  \item \hyperref[user-story:i2-e4-u3]{I2-E4-U3}: Digitalizzazione del catalogo Mambelli
        \begin{itemize}
          \item Ottenere il precedente catalogo (effort 1)
          \item Stabilire uno schema per il database dove immagazzinarlo (effort 4)
          \item Implementare il database (effort 5)
        \end{itemize}
  \item \hyperref[user-story:i1-e1-u1]{I1-E1-U1}: Soddisfazione degli ordini nel piano di produzione generato
        \begin{itemize}
          \item Raccogliere i dati forniti dal committente (effort 2)
          \item Analisi dei dati (effort 10)
          \item Pulizia dei dati (effort 5)
          \item Definire i criteri di valutazione (effort 2)
          \item Definizione di un modello (effort 7)
          \item Training del modello (effort 3)
          \item Valutazione delle prestazioni del modello (effort 4)
          \item Messa in produzione (effort 3)
        \end{itemize}
  \item \hyperref[user-story:i1-e2-u1]{I1-E2-U1}: Visualizzazione del volume di produzione pianificato
        \begin{itemize}
          \item Definizione dell'interfaccia che il risultato del modello deve seguire (effort 2)
          \item Realizzare un mockup del risultato finale (effort 4)
          \item Richiedere la validazione del cliente del mockup (effort 3)
          \item Integrare il mockup nell'applicativo per la pianificazione (effort 6)
        \end{itemize}
  \item \hyperref[user-story:i1-e2-u2]{I1-E2-U2}: Visualizzazione del volume di produzione effettivo
        \begin{itemize}
          \item Realizzare un mockup del risultato finale (effort 4)
          \item Richiedere la validazione del cliente del mockup (effort 3)
          \item Integrare il mockup nell'applicativo per la pianificazione (effort 6)
        \end{itemize}
  \item \hyperref[user-story:i1-e2-u3]{I1-E2-U3}: Filtro nella visualizzazione del piano di produzione
        \begin{itemize}
          \item Realizzare un mockup del risultato finale (effort 4)
          \item Richiedere la validazione del cliente del mockup (effort 3)
          \item Integrare il mockup nell'applicativo per la pianificazione (effort 6)
        \end{itemize}
  \item \hyperref[user-story:i1-u1]{I1-U1}: Modifica del piano di produzione
        \begin{itemize}
          \item Stabilire le modalità con cui interagire con il piano di produzione realizzato (effort 3)
          \item Implementare le query di modifica del piano (effort 2)
          \item Integrazione nell'interfaccia utente della possibilità di modificare il piano (effort 5)
        \end{itemize}
\end{itemize}
